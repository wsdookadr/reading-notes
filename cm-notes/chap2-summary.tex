\documentclass[10pt]{article}
\usepackage{amssymb,amsmath,amsthm,amsfonts}
\usepackage{multicol,multirow}
\usepackage{spverbatim}
\usepackage{amssymb,amsmath,amsthm,amsfonts}
\usepackage{multicol,multirow}
\usepackage{calc}
\usepackage{ifthen}
\usepackage{amsmath,amscd}
\usepackage{tikz-cd}
%Note: tikz-cd docs https://ctan.math.washington.edu/tex-archive/graphics/pgf/contrib/tikz-cd/tikz-cd-doc.pdf

%Note: to break to the next column do
%\vfill\null
%\columnbreak
\usepackage[landscape]{geometry}
\usepackage[colorlinks=true,citecolor=blue,linkcolor=blue]{hyperref}
\usepackage{titlesec}

\usepackage{pict2e}

% Note: http://mirrors.ibiblio.org/CTAN/macros/latex/contrib/pict2e/pict2e.pdf
%
% Description: Custom symbols designed for compact interpretation of identities
%
% Lower triangle filled or empty
% Upper triangle filled or empty
% Filled square
% Diagonal

\DeclareRobustCommand{\LR}{%
  \begingroup
  \setlength{\unitlength}{1ex}%
  \begin{picture}(2.5,2.5)
  \polyline(0,1.7)(1.7,0)(0,0)(0,1.7)
  \end{picture}%
  \endgroup
}

\DeclareRobustCommand{\LRF}{%
  \begingroup
  \setlength{\unitlength}{1ex}%
  \begin{picture}(2.5,2.5)
  \polygon*(0,1.7)(1.7,0)(0,0)(0,1.7)
  \end{picture}%
  \endgroup
}

\DeclareRobustCommand{\UR}{%
  \begingroup
  \setlength{\unitlength}{1ex}%
  \begin{picture}(2.5,2.5)
  \polyline(0,1.7)(1.7,1.7)(1.7,0)(0,1.7)
  \end{picture}%
  \endgroup
}

\DeclareRobustCommand{\URF}{%
  \begingroup
  \setlength{\unitlength}{1ex}%
  \begin{picture}(2.5,2.5)
  \polygon*(0,1.7)(1.7,1.7)(1.7,0)(0,1.7)
  \end{picture}%
  \endgroup
}

\DeclareRobustCommand{\SWH}{%
  \begingroup
  \setlength{\unitlength}{1ex}%
  \begin{picture}(2.5,2.5)
  \polygon(0,0)(1.7,0)(1.7,1.7)(0,1.7)(0,0)
  \end{picture}%
  \endgroup
}

\DeclareRobustCommand{\SBL}{%
  \begingroup
  \setlength{\unitlength}{1ex}%
  \begin{picture}(2.5,2.5)
  \polygon*(0,0)(1.7,0)(1.7,1.7)(0,1.7)(0,0)
  \end{picture}%
  \endgroup
}

\DeclareRobustCommand{\DIAG}{%
  \begingroup
  \setlength{\unitlength}{1ex}%
  \begin{picture}(2.5,2.5)
  \polyline(0,1.7)(1.7,0)
  \end{picture}%
  \endgroup
}

% rising/falling factorial
\newcommand{\ffac}[1]{%
  x^{\underline{#1}}%
}
\newcommand{\ffact}[2]{% 
  {#1}^{\underline{#2}}%
}
\newcommand{\rfac}[1]{%
  x^{\overline{#1}}%
}
\newcommand{\rfact}[2]{% 
  {#1}^{\overline{#2}}%
}


% Note: document width x height
% https://www.overleaf.com/learn/latex/Page_size_and_margins
% https://www.overleaf.com/learn/latex/Lengths_in_LaTeX
\paperwidth = 17.6in
\paperheight = 8in
\topmargin = -1.1in
\setlength{\textheight}{0.95 \paperheight}
\setlength{\linewidth}{0.9 \paperwidth}
\setlength{\textwidth}{0.95 \paperwidth}

\headheight = 0pt
\voffset = 0pt
\hoffset = -1.5in
\marginparwidth = 0pt
\pagestyle{empty}
% title space before/after and font size
\titlespacing\section{0pt}{0pt}{-3pt}
\titlespacing\subsection{0pt}{0pt}{-3pt}
\titleformat{\section}{\normalfont\fontsize{10}{12}\bfseries}{\thesection}{1em}{}
\titleformat{\subsection}{\normalfont\fontsize{9}{12}\bfseries}{\thesection}{1em}{}

\setlength{\columnseprule}{0.4pt}
\setlength{\columnsep}{0cm}
\makeatletter
\makeatother
\setcounter{secnumdepth}{0}
\setlength{\parindent}{0pt}
\setlength{\parskip}{0pt}



\title{Chapter 2 - Summary}
\begin{document}
\raggedright
\footnotesize
\begin{center}
     \Large{\textbf{CM summary - chapter 2}} \\
\end{center}

\begin{multicols}{3}
\section{Terminology}
\begin{array}{ll}
H_n & \text{harmonic number} \\
    & H_n = \sum_{k=1}^n \frac{1}{k} = 1 + \frac{1}{2} + \cdots + \frac{1}{n} \\
    & H_0 = 0 \\
\Delta & \text{difference operator} \\ 
 & \Delta f(x) = f(x+1) - f(x) \\
E & \text{shift operator} \\
  & E f(x) = f(x+1) \\
D & \text{derivative operator} \\
\sum f(x) \delta x & \text{indefinite sum} \\ 
\sum_a^b f(x) \delta x & \text{definite sum} \\
\sum & \text{antidifference operator} \\
 & \Delta f(n) = g(n) \iff \sum g(n) \delta n = f(n) \\
 & \Delta f(n) = g(n) \implies \sum_{a}^b g(n) = f(b) - f(a) \\
 & \Delta f(n) = g(n) \implies \sum_a^b g(x) \delta x = -\sum_b^a g(x) \delta x  \\
\end{array}

\section{Finite calculus analogies with integral calculus}
{
\everymath{\displaystyle}
\begin{tikzcd}
\Delta f(x) \arrow[r,leftrightarrow] \arrow[d, leftrightarrow] & \sum f(x) \delta x  \arrow[d, leftrightarrow] \\
D f(x) \arrow[r,leftrightarrow] & \int f(x) dx
\end{tikzcd}
}
\\
\vspace{0.3cm}
In the above $\delta x = 1$ (\href{https://math.stackexchange.com/a/235691}{reference})
\section{Falling/Rising factorial identities}


{
\everymath{\displaystyle}
Assume $m > 0$
\begin{multicols}{2}
\begin{array}{ll}
\ffac{-m} & = \frac{1}{(x+1)\cdots (x+m)}\\
\ffac{m} & = (x-m+1) \cdots (x-1) x\\
\ffac{+2} & = x(x-1)\\
\ffac{+1} & = x\\
\ffac{0} & = 1\\
\ffac{-1} & = \frac{1}{x+1}\\
\ffac{-2} & = \frac{1}{(x+1)(x+2)}\\
\end{array}
\vfill\null
\columnbreak
\begin{array}{l}
\ffac{m+1} = \ffac{m} \ffact{(x-m)}{n} \text{  (law of exponents)}\\
\rfac{-m} = \frac{1}{\rfact{(x-m)}{m}} = \frac{1}{\ffact{(x-1)}{m}} \\
\ffact{(x+y)}{m} \text{ analog of binomial theorem applies}\\
\frac{\ffact{x}{m}}{\ffact{(x-n)}{m}} = \frac{\ffact{x}{n}}{\ffact{(x-m)}{n}} \\
\rfact{x}{m} = (-1)^m\ffact{(-x)}{m} = \ffact{(x+m-1)}{m} = \frac{1}{\ffact{(x-1)}{-m}}\\
\ffact{x}{m} = (-1)^m\rfact{(-x)}{m} = \rfact{(x-m+1)}{m} = \frac{1}{\rfact{(x+1)}{-m}}\\
\end{array}
\begin{array}{ll}

\end{array}
\end{multicols}
}

\section{Summation properties}
{\everymath{\displaystyle}

Double-counting: $[k \in K] + [k\in K'] = [k\in K \cap K'] + [k\in K \cup K']$\\
Interchanging order of summation: $\sum_{j} \sum_{k} a_{j,k} [P(j,k)] = \sum_{P(j,k)} a_{j,k} = \sum_{k} \sum_{j} a_{j,k} [P(j,k)]$
Interchanging summation order \textbf{vanilla version}: $\sum_{j\in J} \sum_{k\in K} a_{j,k} = \sum_{\substack{j\in J\\k\in K}} a_{j,k} = \sum_{k\in K} \sum_{j\in J} a_{j,k}$\\
Interchanging summation order \textbf{rocky road version}: $\sum_{j\in J} \sum_{k\in K(j)} a_{j,k} = \sum_{k\in K'} \sum_{j\in J'(k)} a_{j,k}$\\
where the following holds: $[j\in J][k\in K(j)] = [k\in K'][j\in J'(k)]$\\
Application of rocky road: 
$[1\leq j \leq n] [j\leq k \leq n] = [1\leq j \leq k \leq n] = [1 \leq k \leq n] [1\leq j \leq k]$\\
$\sum_{j=1}^{n} \sum_{k=j}^n a_{j,k} = \sum_{1\leq j\leq k\leq n} a_{j,k} =\sum_{k=1}^{n} \sum_{j=1}^k a_{j,k}$
}

\vfill\null
\columnbreak

\section{Iverson bracket properties}

$\URF + \LRF = \SBL + \DIAG$

$ [1\leq j\leq k\leq n] + [1\leq k\leq j \leq n] = [1\leq j,k \leq n] + [1\leq j=k \leq n] $

$\UR + \LR = \SBL - \DIAG$

$ [1\leq j < k \leq n] + [1\leq k < j \leq n] = [1\leq j,k\leq n] - [1\leq j=k\leq n] $

\section{Summation by parts}
{\everymath{\displaystyle}
\begin{array}{ll}
\text{Difference of a product} & \Delta(u\cdot v) = u\Delta v + E\ v\Delta u\\
\text{Summation by parts}      & \sum u\cdot \Delta v = u v - \sum E\ v\Delta u\\
& E \text{ is applied only to the term immediately after it} 
\end{array}
}

\section{Finite difference table}
% array is already in math mode
{
\everymath{\displaystyle}
\bgroup
\def\arraystretch{1.6}
\begin{array}{|r|l|l|}
\hline
\mathbf{f(x)} & \mathbf{\Delta f(x)} & \textbf{note}\\
\hline
\ffac{m} & m\cdot \ffac{m-1} & \\ 
\hline
H_x & \frac{1}{x+1} = \ffac{-1}& \\ 
\hline
c^x & (c-1)c^x & \\ 
\hline
\frac{\ffac{m+1}}{m+1} & \ffac{m} & m \neq -1 \\ 
\hline
c \cdot u & c\cdot \Delta u & \\ 
\hline
u + v & \Delta u + \Delta v & \\ 
\hline
\frac{c^x}{c-1} & c^x & \\
\hline
2^x & 2^x & \\
\hline
\ffact{c}{x} & \frac{\ffact{c}{x+2}}{c-x} & \\
\hline
\end{array}
\egroup
}
\\
(see table 55 in the book)
\section{Sum/Product properties}
{
\everymath{\displaystyle}
\bgroup
\renewcommand{\arraystretch}{2}
\begin{array}{|l|l|l|}
\hline
\text{sum law} & \text{product law} & \text{name}\\
\hline
\sum_{k\in K} c a_k = c\sum_{k\in K} a_k & \prod_{k\in K} a_k^c = \left(\prod_{k\in K} a_k \right)^c & \text{distributive law (2.15)}\\
\hline
\sum_{k\in K} a_k + b_k = \sum_{k\in K} a_k + \sum_{k\in K} b_k & \prod_{k\in K} a_k b_k = \prod_{k\in K} a_k \cdot \prod_{k\in K} b_k & \text{associative law(2.16)} \\
\hline
\sum_{\substack{j\in J\\k\in K}} a_j b_k = \left(\sum_{j\in J} a_j\right) \cdot \left(\sum_{k\in K} b_k \right) & & \text{general distributive law (2.28)}\\
\hline
\sum_{k\in K} a_k = \sum_{k\in K} a_{p(k)} & \prod_{k\in K} a_k = \prod_{k\in K} a_{p(k)} & \text{commutative law(2.17)} \\
\hline
\sum_{\substack{k\in K\\j\in J}} a_{j,k} = \sum_{k\in K}\sum_{j\in J} a_{j,k} & \prod_{\substack{k\in K\\j\in J}} a_{j,k} = \prod_{k\in K}\prod_{j\in J} a_{j,k} & \\ 
\hline
\sum_{k\in K} a_k = \sum_{k} a_k \cdot [k\in K] & \prod_{k\in K} a_k = \prod_{k} a_k^{[k\in K]} & \\
\hline
\sum_{k\in K} 1 = |K| & \prod_{k\in K} c = c^{|K|} & \\
\hline
\end{array}
\egroup
}

\vfill\null
\columnbreak

\section{General techniques for recurrences or sums}
\subsection{Summation factor}
{\everymath{\displaystyle}
Recurrence type: $a_n T_n = b_n T_{n-1} + c_n$

Step 1. Multiply both sides by $s_n=\frac{a_{n-1}a_{n-2}\cdots a_1}{b_n b_{n-1}\cdots b_2}$

Note: $s_n b_n = s_{n-1}a_{n-1}$

Step 2. Build $S_n = S_{n-1} + s_n c_n$

Step 3. $S_n = s_0 a_0 T_0 + \sum{k=1}^n s_k c_k = s_1b_1T_0 \sum_{k=1}^n s_k c_k$

Step 4. Closed form: $T_n = \frac{1}{s_n a_n}\left(s_1b_1T_0 + \sum_{k=1}^n s_k c_k\right)$
}

\subsection{Repertoire method}
{\everymath{\displaystyle}
We have a recurrence $R_n$ with an initial condition and a recurrence relation.

Step 1. Write the general form $R_n = A(n)\alpha + B(n) \beta + C(n) \delta + D(n) \gamma$

Step 2. Set $R_n$ to be $1,n,n^2,n^3,\dots$ (the repertoire) successively and determine $\alpha,\beta,\delta,\gamma$ for each.

Step 3. Build a system of linear equations in $A(n),B(n),C(n),D(n)$ where the right-hand side will be the functions from the repertoire. 

Step 4. Solve the system to determine the functions $A,B,C,D$ and thereby finding a closed form for $R_n$

}

\subsection{Perturbation method/scheme}
{\everymath{\displaystyle}

Step 1. Rewrite the sum $S_{n+1} = \sum_{0\leq k \leq n+1} a_k $ by splitting off the last and first term:

$S_{n+1} = S_n + a_{n+1}$

$S_{n+1} = a_0 + \sum_{1\leq k \leq n+1} a_k = a_0 + \sum_{0\leq k \leq n} a_{k+1}$

Step 2. Make the last sum look like $S_n$.

Step 3. Use these two expressions to find a closed form for $S_n$

}

\subsection{Replace the sum by integrals}
{\everymath{\displaystyle}

Step 1. Replace the sum $S_n=\sum_{k=1}^n f(k)$ by an integral $I_n=\int_{0}^n f(x) dx$, solve the integral.

Step 2. Look at the error in the approximation $E_n = S_n - I_n$ , find a closed form for it which leads to a closed form for $S_n$. 
}

\newpage

\end{multicols}
\end{document}