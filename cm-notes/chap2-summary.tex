% \documentclass[10pt,landscape,letterpaper]{article}
\documentclass[10pt,landscape]{article}
\usepackage{amssymb,amsmath,amsthm,amsfonts}
\usepackage{multicol,multirow}
\usepackage{spverbatim}
\usepackage{amssymb,amsmath,amsthm,amsfonts}
\usepackage{multicol,multirow}
\usepackage{calc}
\usepackage{ifthen}
\usepackage{amsmath,amscd}
\usepackage{tikz-cd}
% tikz-cd docs https://ctan.math.washington.edu/tex-archive/graphics/pgf/contrib/tikz-cd/tikz-cd-doc.pdf
\usepackage[landscape]{geometry}
\usepackage[colorlinks=true,citecolor=blue,linkcolor=blue]{hyperref}
\ifthenelse{\lengthtest { \paperwidth = 11in}}
    { \geometry{top=.5in,left=.5in,right=.5in,bottom=.5in} }
	{\ifthenelse{ \lengthtest{ \paperwidth = 297mm}}
		{\geometry{top=1cm,left=1cm,right=1cm,bottom=1cm} }
		{\geometry{top=1cm,left=1cm,right=1cm,bottom=1cm} }
	}
\pagestyle{empty}
\makeatletter
\renewcommand{\section}{\@startsection{section}{1}{0mm}%
                                {-1ex plus -.5ex minus -.2ex}%
                                {0.5ex plus .2ex}%x
                                {\normalfont\large\bfseries}}
\renewcommand{\subsection}{\@startsection{subsection}{2}{0mm}%
                                {-1explus -.5ex minus -.2ex}%
                                {0.5ex plus .2ex}%
                                {\normalfont\normalsize\bfseries}}
\renewcommand{\subsubsection}{\@startsection{subsubsection}{3}{0mm}%
                                {-1ex plus -.5ex minus -.2ex}%
                                {1ex plus .2ex}%
                                {\normalfont\small\bfseries}}
\makeatother
\setcounter{secnumdepth}{0}
\setlength{\parindent}{0pt}
\setlength{\parskip}{0pt plus 0.5ex}
\title{Chapter 2 - Summary}
\begin{document}
\raggedright
\footnotesize
\begin{center}
     \Large{\textbf{CM summary - chapter 2}} \\
\end{center}
\begin{multicols}{2}
\setlength{\premulticols}{1pt}
\setlength{\postmulticols}{1pt}
\setlength{\multicolsep}{1pt}
\setlength{\columnsep}{2pt}

\section{Terminology}
\begin{array}{ll}
H_n & \text{harmonic number} \\
    & H_n = \sum_{k=1}^n \frac{1}{k} = 1 + \frac{1}{2} + \cdots + \frac{1}{n} \\
    & H_0 = 0 \\
\Delta & \text{difference operator} \\ 
 & \Delta f(x) = f(x+1) - f(x) \\
E & \text{shift operator} \\
  & E f(x) = f(x+1) \\
D & \text{derivative operator} \\
\sum f(x) \delta x & \text{indefinite sum} \\ 
\sum_a^b f(x) \delta x & \text{definite sum} \\
\sum & \text{antidifference operator} \\
 & \Delta f(n) = g(n) \iff \sum g(n) \delta n = f(n) \\
 & \Delta f(n) = g(n) \implies \sum_{a}^b g(n) = f(b) - f(a) \\
 & \Delta f(n) = g(n) \implies \sum_a^b g(x) \delta x = -\sum_b^a g(x) \delta x  \\
\end{array}


\section{Finite calculus analogies with integral calculus}
{\large
\everymath{\displaystyle}
\begin{tikzcd}
\Delta f(x) \arrow[r,leftrightarrow] \arrow[d, leftrightarrow] & \sum f(x) \delta x  \arrow[d, leftrightarrow] \\
D f(x) \arrow[r,leftrightarrow] & \int f(x) dx
\end{tikzcd}
}
\\
\vspace{0.3cm}
In the above $\delta x = 1$ (\href{https://math.stackexchange.com/a/235691}{reference})
\section{Falling/Rising factorial identities}

{\large
\begin{array}{lll}
x^{\underline{-m}} & = \frac{1}{(x+1)\cdots (x+m)} & \text{for } m > 0\\
x^{\underline{m}} & = \frac{x!}{(x-n)!} = (x-n+1)(x-n+2)\cdots (x-1) x & \text{for } m > 0\\
x^{\underline{+2}} & = x(x-1)\\
x^{\underline{+1}} & = x\\
x^{\underline{0}} & = 1\\
x^{\underline{-1}} & = \frac{1}{x+1}\\
x^{\underline{-2}} & = \frac{1}{(x+1)(x+2)}\\
\end{array}
}

\section{Summation by parts}

\begin{array}{ll}
\text{Difference of a product} & \Delta(u\cdot v) = u\Delta v + E\ v\Delta u\\
\text{Summation by parts}      & \sum u\cdot \Delta v = u v - \sum E\ v\Delta u\\
& E \text{ is applied only to the term immediately after it} 
\end{array}

\vfill\null
\columnbreak
\section{Finite difference table}
% array is already in math mode
(table 55 in the book)\\
{\large
\bgroup
\def\arraystretch{1.6}
\begin{array}{|r|l|l|}
\hline
\mathbf{f(x)} & \mathbf{\Delta f(x)} & \textbf{note}\\
\hline
x^{\underline{m}} & m\cdot x^{\underline{m-1}} & \\ 
\hline
H_x & \frac{1}{x+1} = x^{\underline{-1}}& \\ 
\hline
c^x & (c-1)c^x & \\ 
\hline
\frac{x^{\underline{m+1}}}{m+1} & x^{\underline{m}} & m \neq -1 \\ 
\hline
c \cdot u & c\cdot \Delta u & \\ 
\hline
u + v & \Delta u + \Delta v & \\ 
\hline
\frac{c^x}{c-1} & c^x & \\
\hline
2^x & 2^x & \\
\hline
\end{array}
\egroup
}

\section{Sum/Product properties}

{
\everymath{\displaystyle}
\bgroup
\renewcommand{\arraystretch}{2}
\begin{array}{|l|l|}
\hline
\text{sum property} & \text{product property} \\
\hline
\sum_{k\in K} c a_k = c\sum_{k\in K} a_k & \prod_{k\in K} a_k^c = \left(\prod_{k\in K} a_k \right)^c \\
\hline
\sum_{k\in K} a_k + b_k = \sum_{k\in K} a_k + \sum_{k\in K} b_k & \prod_{k\in K} a_k b_k = \prod_{k\in K} a_k \cdot \prod_{k\in K} b_k \\
\hline
\sum_{\substack{k\in K\\j\in K}} a_{j,k} = \sum_{k\in K}\sum_{j\in J} a_{j,k} & \prod_{\substack{k\in K\\j\in K}} a_{j,k} = \prod_{k\in K}\prod_{j\in J} a_{j,k} \\ 
\hline
\sum_{k\in K} a_k = \sum_{k\in K} a_{p(k)} & \prod_{k\in K} a_k = \prod_{k\in K} a_{p(k)} \\
\hline
\sum_{k\in K} a_k = \sum_{k} a_k \cdot [k\in K] & \prod_{k\in K} a_k = \prod_{k} a_k^{[k\in K]} \\
\hline
\sum_{k\in K} 1 = |K| & \prod_{k\in K} c = c^{|K|}\\
\hline
\end{array}
\egroup
}

\newpage

\end{multicols}
\end{document}