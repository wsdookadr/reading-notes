% \documentclass[12pt,a4paper]{report}
\documentclass[a4paper,10pt,fleqn]{article}
\usepackage{multicol}
\usepackage{amsmath}
\usepackage[colorlinks=true,citecolor=blue,linkcolor=blue]{hyperref}

\setlength{\mathindent}{0cm}
\begin{document}
\newcommand{\ffac}[1]{%
  x^{\underline{#1}}%
}
\newcommand{\ffact}[2]{% 
  {#1}^{\underline{#2}}%
}

\everymath{\displaystyle}

\textbf{Problem 2.26}

Notation: $T = \prod_{1\leq k \leq n} a_k$

Express $P=\prod_{1\leq j \leq k \leq n} a_j a_k$ in terms of $T$.

By symmetry, the indices can be renamed $P=\prod_{1\leq j \leq k \leq n} a_j a_k=\prod_{1\leq k \leq j \leq n} a_j a_k$

Converting the Iverson bracket identity

$ [1\leq j\leq k\leq n] + [1\leq k\leq j \leq n] = [1\leq j,k \leq n] + [1\leq j=k \leq n] $

to products, yields:

$
\begin{array}{rll}
\prod_{1\leq j \leq k \leq n} a_j a_k \cdot \prod_{1\leq k \leq j \leq n} a_j a_k & = &
 \prod_{1\leq j,k \leq n} a_j a_k \cdot \prod_{1\leq k=j \leq n} a_j a_k \\
P^2 & = & \underbrace{\prod_{1\leq j,k \leq n} a_j a_k}_{A} \cdot \underbrace{\prod_{1\leq k=j \leq n} a_j a_k}_{B} \\
\end{array}
$

\begin{multicols}{2}

$
\begin{array}{ll}
A & = \prod_{1\leq j,k \leq n} a_j a_k \\
  & = \prod_{1\leq j\leq n} \prod_{1\leq k\leq n} a_j a_k\\
  & = \prod_{1\leq j\leq n} a_j^n \prod_{1\leq k\leq n} a_k \\
  & = \prod_{1\leq j\leq n} a_j^n \cdot T\\
  & = T^n \prod_{1\leq j\leq n} a_j^n \\
  & = T^n \left(\prod_{1\leq j\leq n} a_j\right)^n \\  
  & = T^n T^n = T^{2n}\\
\end{array}
$

\columnbreak

$
\begin{array}{ll}
B & = \prod_{1\leq j=k \leq n} a_j a_k \\
  & = \prod_{1\leq j \leq n} a_j^2 \\
  & = \left(\prod_{1\leq j \leq n} a_j\right)^2 = T^2 \\
\end{array}
$

\end{multicols}

$P^2 = T^{2(n+1)} \iff P = T^{n+1}$

\end{document}
